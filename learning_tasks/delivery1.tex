\documentclass[12pt]{article}
%\usepackage[norsk]{babel}
\usepackage[utf8]{inputenc}
\usepackage[skip=5mm]{parskip}
\usepackage{geometry} % Marginer
\usepackage{tabularx}
\usepackage{graphicx}
\usepackage[font=small,labelfont=bf]{caption}
\usepackage[dvipsnames]{xcolor}
\usepackage{float}
\usepackage[normalem]{ulem}
\usepackage[T1]{fontenc}
\usepackage{listings}
\usepackage{listingsutf8}
\usepackage{pdfpages}
\usepackage{enumitem}
\usepackage{amssymb}
\usepackage{mathtools}
\usepackage{dirtytalk}
\usepackage{booktabs}
\usepackage{hyperref}

\newcommand{\task}[1]{\vspace*{5mm}\hspace*{-2cm}\fbox{#1}\\[-5mm]}
\newcommand{\subtask}[1]{\hspace*{-1cm}\textbf{(#1)}\\[-5mm]}

\title{TDT4165 Programming Languages\\Scala Project Delivery 1}
\author{\small{Random Group 4}\\ \small{Sondre T Bungum, Håvard R. Krogstie, Hermann Mørkrid}}
\date{\today}

\begin{document}
\maketitle

\task{1}
\textbf{Scala Introduction}

All the code for this task can be found in \verb|task1/Main.scala|.

\subtask{d}
The function \verb|fibonacci(n: Int)| gives the $n$'th fibonacci number,
with $0$ being the zeroth, and $1$ being the first. The rest of the seqence
is implemented using the recursive definition of fibonacci.

The function return type is \verb|BigInt| because the fibonacci sequence
quickly exceeds $2^{31}-1$, which is the maximum value representable as an \verb|Int|.
A \verb|BigInt| however does not have a pre-determined range of possible integer values.
It will allocate enough space to fit the value as it grows.

\task{2}
\textbf{Concurrency in Scala}

\subtask{d}
A deadlock occurs when two threads are waiting for a responce for one another.
This can be avoided by creating order so that we are always sure that an action
is finished or can be finished before or when it is required.
\end{document}
